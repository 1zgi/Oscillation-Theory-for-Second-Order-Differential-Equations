documentclass[12pt]{artcle}
 usepackage{amsmath,amssymb,latexsym, graphcx}
 usepackage{amsthm}
 newcommand{ds}{dsplaystyle}
 newtheorem{teo}{Theorem}
 newtheorem{dfn}{Defnton}
 newtheorem{ex}{Example}
 begn{document}
 begn{ttlepage} 
newcommand{HRule}{rule{lnewdth}{0.5mm}} 
center 
ncludegraphcs[wdth=4cm]{IMG_0058}
  [0.5cm] 
textsc{LARGE bf c{C}ankaya Unversty}[1.5cm] 
textsc{Large Department of Mathematcs}[0.5cm] 
textsc{large Graduaton Project}[0.5cm] 
HRule[1cm]
 {hugebfseres Oscllaton Theory for Second Order Dfferental Equatons}[0.4cm] 
HRule[1.5cm]
 begn{mnpage}{0.4textwdth}
 begn{flushleft}
 large
 textt{Student(Author)}
 İzg textsc{KANATLI} 
end{flushleft}
 end{mnpage}
 ~
 begn{mnpage}{0.4textwdth}
begn{flushrght}
 large
 textt{Supervsor}
 Assoc. Prof. Dr. {Ekn} textsc{UĞURLU} 
end{flushrght}
 end{mnpage}
 end{ttlepage} 
secton{Introducton}
 ndentpar A Sturm-Louvlle dfferental equaton s sad to be oscllatory at $nfty$, f there 
exsts a nontrval soluton of ths equaton whch havng nfntly many zeros n every nterval of 
the form $[alpha,nfty)$. Otherwse, t sad to be nonoscllatory at $nfty$. As we shall see that 
dsconjugacy and admssble functons are some mportant concepts related wth ths theory. If 
no soluton of a dfferental equaton has more than one zero n nterval I then t s sad to be 
dsconjugate on I. If a functon and ts frst dervatve are contnuous on some nterval, and the 
functon attans the value zero at both ends of the nterval then t s sad to be admssble. 
par Let us begn wth the smple wave equaton
   begn{equaton}
 y''(x) + k^2y(x) = 0.%equaton 1
   end{equaton}
 Ths equaton has the solutons,
 begn{equaton} 
C(sn k)(x-x_0) 
 end{equaton}
 or
 begn{equaton}
 C(cos k)(x-x_0),
 end{equaton}
 nondent and clearly these solutons are oscllatory n $mathbb{R}$, where $k$ s a postve 
constant and $C$ s an arbtrary constant. It can be seen that the zeros of lnearly ndependent 
solutons of the equaton (1) are nterlaced. Equaton (1) can be consdered as the specal case of 
the followng equaton,
 begn{equaton} 
y''(x) +q(x)y(x)=0, %equaton 2
 end{equaton}
 where $q(x)$ s contnuous on some nterval. In general as we all see that, f $y_1(x)$ and 
$y_2(x)$ are two lnearly ndependent solutons of (1) and $x_1$ and $x_2$ are consecutve 
zeros of $y_1(x)$, then $y_2(x)$ has one and only one zero n $(x_1,x_2)$.
ndent It s known that a soluton $y(x)$ of the equaton (2) s contnuous on the gven nterval 
whenever $q(x)$ s at least contnuous on that nterval. Accordng to the theory of ordnary 
dfferental equatons a soluton $y(x)$ of the equaton (2) satsfyng the ntal condtons $y(x_0) 
= 0$ and $y'(x_0) = 0$ should be a trval soluton. Therefore, nothng can be sad about the 
zeros of such trval solutons. So, the zeros of nontrval solutons are needed to the examned.
 ndent It s better to remnd that a pont $x = x_1$ s called a zero of the soluton $y(x)$ of (2) f 
$y(x_1) = 0$, and accordng to prevous explanaton $y'(x_1)$ should be dfferent from zero snce 
otherwse t becomes a trval soluton. From the theory of calculus, t s known that f 
$y'(x_1)=alpha  0$ then there exsts an nterval $(x_1-h_1,x_1+h_2)$ n the doman of $y(x)$ 
such that, $y(x)$ s negatve (postve) on $(x_1-h_1,x_1)$ and postve (negatve) on $
 (x_1,x_1+h_2)$, where $h_1$ and $h_2$ are some postve constants. The ponts $x_1$ and 
$x_2$ are called consecutve zeros of the soluton $y(x)$ of (2) f $y(x_1) = y(x_2) = 0$ and 
there s no other zero between $x_1$ and $x_2$, where $x_1x_2$.
 ndent We shall note that some Sturm-Louvlle equatons may have the solutons so that these 
solutons may not have any zeros on the gven nterval. Indeed, the followng second order 
Sturm-Louvlle equaton
 begn{equaton}
 y''(x)-k^2y(x)=0
 end{equaton} 
has the solutons
 begn{equaton}
 Ce^{kx}
 end{equaton}
 and
 begn{equaton}
 Ce^{-kx},
 end{equaton}
 where $C$ and $k$ are some arbtrary real numbers, and as can be seen that these solutons do 
not have any zero on $mathbb{R}$, unless $C=0$.
 secton{Sturm's Seperaton Theorem}%Secton 2
 ndentpar The frst fundamental theorem on comparson of the zeros of the Sturm-Louvlle 
equaton has been gven by Sturm. Frst of all we shall share the followng seperaton theorem.
 [baselneskp]
 {bf Theorem 2.1}textt { Let $y_1(x)$ and $y_2(x)$ be two lnearly ndependent solutons of the 
equaton (2). If $x_1$ and $x_2$ are consecutve zeros of $y_1(x)$ then $y_2(x)$ has one and 
only one zero n $(x_1,x_2)$.}[baselneskp]
 {bf Proof} Assume that $y_1(x)0$ for $x_1xx_2$ such that $y'_1(x_1)0$. Accordng to the 
assumpton $x_1$ and $x_2$ are consecutve zeros of $y(x)$ so that $y_1'(x_2)0$. Therefore, 
$y_1(x)$ s ncreasng at $x=x_1$ and $y_1(x)$ s decreasng at $x=x_2$. It s known that the 
Wronskan of $y_1(x)$ and $y_2(x)$ s a non-zero constant. Here the Wronskan $W[y_1,y_2]$ of 
the solutons $y_1$ and $y_2$ s defned by
 begn{equaton}nonumber
 W[y_1,y_2] = begn{vmatrx} y_1 & y_2  y'_1 & y'_2 end{vmatrx}=  y_1y'_2-y_2y'_1.
 end{equaton}
 So, at  $x=x_1$ we havebegn{equaton}nonumber y_1(x_1)y'_2(x_1)-y'_1(x_1)y_2(x_1) = -y'_1(x_1)y_2(x_1).
 end{equaton}
 Smlarly for $x=x_2$, begn{equaton}nonumber y_1(x_2)y'_2(x_2)-y'_1(x_2)y_2(x_2) = -y'_1(x_2)y_2(x_2)
 end{equaton}
 s obtaned. So
 begn{equaton}nonumber 
y_2(x_1)y'_1(x_1)=y_2(x_2)y'_1(x_2),
 end{equaton}
 and hence, begn{equaton}nonumber (sgn)y_2(x_1)=-(sgn)y_2(x_2).
 end{equaton} Ths mples that $y_2(x)$ has at least one zero n $(x_1,x_2)$.
 ndent If there were two zeros $x_1$ and $x_2$ of $y_2(x)$ between $(x_1,x_2)$, $y_1(x)$ 
would have at least one zero n $(x_1,x_2)$. Ths contradcts the assumpson of $x_1,x_2$ 
beng consecutve zeros of $y_1(x)$. Ths completes the proof. $blacksquare$
 %newpage
 secton{Comparson Theorems}%Secton 3
 ndentpar We shall ntroduce some results on the zeros of solutons of some Sturm-Louvlle 
equatons.
 subsecton{Sturm's Comparson Theorem} %subsecton 1
 ndentpar Followng theorem s due to Sturm [2].[baselneskp]
 {bf Theorem 3.1.1} textt{Consder two equatons,} %Theorem 3.1.1
 begn{equaton}
 u''(x)+F(x)u(x)=0,
 end{equaton}
 begn{equaton}
 v''(x)+G(x)v(x)=0,
 end{equaton}
 textt{where $F(x),G(x)$ are postve, contnuous and $G(x) ge F(x)$ n $(a,b)$. Let the soluton 
$u(x)$ of the equaton (3) have two consecutve zeros, $x_1$ and $x_2$, $ax_1x_2b$, and 
$v(x)$ be soluton of the equaton (4) wth a zero at $x_1$. Then $v(x)$ has at least a zero $x_3$ 
n $(x_1,x_2)$.}[baselneskp] %Proof startng
 {bf Proof} Frstly, let us multply the equaton (3) by $v(x)$ and multply the equaton (4) by 
$u(x)$. Then we shall subtract the frst equaton from the second equaton so that
 begn{equaton}
v''(x)u(x)-u''(x)v(x)+G(x)v(x)u(x)-F(x)v(x)u(x)=0.
 end {equaton}
 Equaton (5) can also be wrtten as
 begn{equaton}
 tfrac{d}{dx}W[u,v]+(G(x)-F(x))u(x)v(x)=0.
 end {equaton}
 Now, ntegratng the equaton (6) on $(x_1,x_2)$ one has
 begn{equaton}nonumber
 W[u,v](x)bgg ^{x_2}_{x_1}+nt^{x_2}_{x_1}(G(x)-F(x))u(x)v(x) dx=0,
 end {equaton}
 or
 begn{algn}%equaton 7
 nonumber
 u(x_2)v'(x_2)-v(x_2)u'(x_2)-u(x_1)v'(x_1)+v(x_1)u'(x_1) + nt^{x_2}_{x_1}(G(x)-F(x))u(x)v(x) 
dx=0.
 end {algn}
 Hence
 begn{equaton}nonumber-v(x_2)u'(x_2)+ nt^{x_2}_{x_1}(G(x)-F(x))u(x)v(x) dx=0.
 end {equaton}
 If $v'(x_1)0$ and $v(x)$ were postve n nterval $(x_1,x_2)$ then the ntegral would be postve 
and t can be nferred that
 begn{equaton}-v(x_2)u'(x_2) ge 0.
 end{equaton}
 ndent Ths expresson shows that $v(x)$ cannot keep a constant sgn throughout the nterval $
 (x_1,x_2)$. $blacksquare$[baselneskp]%qedsymbol{}
 %
 {bf Theorem 3.1.2}textt{ Let $u(x)$ and $v(x)$ be the solutons of (3) and (4), respectvely, 
such that} %Theorem 3.1.2
 begn{gather}
 u(x_1)=v(x_1)=0 quad ,quad u'(x_1)=v'(x_1)0.
 end{gather}
 textt{Suppose that $u(x)$ s ncreasng n $[x_1,x_2]$ and reaches a maxmum at $x=x_2$. 
Then $v(x)$ reaches a maxmum at some pont $x_3$ wth $x_1x_3x_2$.}[baselneskp]
 {bf Proof} Assume that $u(x)$ reaches a maxmum pont at $x=x_2$. Ths means that 
$u'(x_2)=0$.
 Then (7) can be wrtten as 
begn{equaton}
 u(x_2)v'(x_2)+ nt^{x_2}_{x_1}(G(x)-F(x))u(x)v(x) dx=0.
 end {equaton}
Ths equaton shows that $v'(x_2)0$ f $v(x)0$ on $(x_1,x_2)$. So there should be a maxmum 
of $v(x)$ at $x_3$ on $x_1xx_2$. Ths completes the proof.hspace{8.6cm}$blacksquare$ 
[baselneskp]
 {bf Remark} It s possble to consder much more general verson of the Sturm-Louvlle 
equatons rather than the form gven n (2) as follows. However n many applcatons the sgn of 
the leadng functon of the Sturm-Louvlle equaton s -1. So we shall consder the followng 
equatons
 begn{equaton}%equaton 8-(p_1(x)u'(x))'+p_0(x)u(x)=0,
 end {equaton}
 begn{equaton}%equaton 9-(q_1(x)v'(x))'+q_0(x)v(x)=0,
 end {equaton}
 where $p_10$, $q_10$ and $p_0$, $q_0$ are some real-valued functons on an nterval $
 (x_1,x_2)$.
 subsecton {Sturm-Pcone Comparson Theorem}%SubSecton 2
 ndentpar The restrcton $p_1(x)=q_1(x)$ has been removed by M.Pcone n 1909.
 [baselneskp]
 {bf Theorem 3.2.1} textt{Suppose that $p_1(x) ge q_1(x)$ and $p_0(x) ge q_0(x)$ on an 
nterval $[x_1,x_2]$. If $x_1$ and $x_2$ are consecutve zeros of $u(x)$ of (8) then there should 
exsts at least one zero of every soluton $v(x)$ of (9).}[baselneskp]
 {bf Proof} Consder that $v(x)ne0$ for $xn[x_1,x_2]$. Followng the equaton s known as 
Pcone's dentty;
 begn{algn}%Equaton 10
 tfrac{d}{dx}bgg[frac{u}{v}(vp_1u'-uq_1v')bgg]= &tfrac{d}{dx}bgg[up_1u'-frac{u^2q_1v'}{v}
 bgg]nonumber
 =& (u')^2p_1+(p_1u')'u-Bgg[frac{2uu'v-v'u^2}{v^2}(v'q_1)+ frac{u^2}{v}(q_1v')'Bgg]
 nonumber
 =& (u')^2p_1+p_0u^2+frac{q_1}{v^2}[uv'-vu']^2- q_1(u')^2-u^2q_0nonumber
 =& (u')^2(p_1-q_1)+u^2(p_0-q_0)+frac{q_1}{v^2}[uv'-vu']^2.
 end{algn}
 Integratng both sdes of (10) on $[x_1,x_2]$ we have,
 begn{algn}
 &frac{u}{v}bgg[vp_1u'-uq_1v'bgg]bgg^{x_2}_{x_1}=0
 &= nt^{x_2}_{x_1}bgg((p_1-q_1)(u')^2+(p_0-q_0)u^2+frac{q_1}{v^2}[uv'-vu']^2 bgg)dx  
&ge nt^{x_2}_{x_1}bgg((p_1-q_1)(u')^2+(p_0-q_0)u^2bgg)dx
 end{algn}
 whch s a contradcton unless $p_1 equv q_1$, $p_0 equv q_0$ and $frac{uv'-vu'}{v^2}
 =(frac{-u}{v})'=0$. However ths mples that $frac{u}{v}=$ constant. Consequently, $v(x_1) = 
v(x_2)=0$. Ths s a contradcton. $blacksquare$
 subsecton{Leghton's Comparson Theorem}
{bf Theorem 3.3.1} textt{If $nt^{nfty}_{alpha}frac{1}{p_1}dx = nfty$ and $nt^{nfty}
 _{alpha}{p_0}dx = -nfty$ then the Sturm-Louvlle equaton (8) s oscllatory at $nfty$, where 
$alpha$ s a fxed constant.}[baselneskp]
 {bf Proof} Suppose that $h = -frac{p_1u'}{u}$. Then t s obtaned that
 begn{algn}
 h'=-bgg(frac{(p_1u')'u-u'(p_1u')}{u^2}bgg) = &frac{-p_0u^2+u'^2p_1}{u^2}
 =& -p_0+h^2frac{1}{p_1}=h'.
 end{algn}
 If $u0$ on $[beta,nfty)$, $alphalebetalenfty$, the equaton (8) s nonoscllatory at $
 nfty$. Integratng $h'$ on $[beta,nfty)$ t s obtaned that
 begn{algn}
 h(x)-h(beta) =& -nt^{x}_{beta} p_0(t) dt + nt^{x}_{beta}frac{h^2(t)}{p_1(t)}dt,
 end{algn}
 and
 begn{algn}
 h(x) = & h(beta)-nt^{x}_{beta}p_0(t)dt + nt^{x}_{beta}frac{h^2(t)}{p_1(t)}dt.
 end{algn}
 Adoptng the notaton begn {equaton×} g(x)=nt^{x}_{beta}frac{h^2(t)}{p_1(t)}dt, end 
{equaton} one gets that $h(x)g(x)$ on $(beta,nfty)$. Takng the dervatve of $g(x)$ t s 
obtaned that
 begn{equaton}
 g'(x)geqfrac{h^2(x)}{p_1(x)}.
 end{equaton}
 It can be wrtten as
 begn{equaton}
 frac{h^2(x)}{p_1(x)}frac{g^2(x)}{p_1(x)},
 end{equaton}
 or
 begn{equaton}
 frac{g'(x)}{g^2(x)}frac{1}{p_1(x)}.
 end{equaton}
 Integratng both sdes of (11) on $[gamma,nfty)$, one obtaned that
 begn{algn}
 nt^{nfty}_{gamma}frac{g'(x)}{g^2(x)}dx & nt^{nfty}_{gamma}frac{1}{p_1(x)}dx
 end{algn}
 whch gves
 begn{algn}
 nt^{nfty}_{gamma} u^{-2}du = -u^{-1}bgg^{nfty}_{gamma} = -frac{1}{g}bgg^{nfty}
 _{gamma} nt^{nfty}_{gamma}frac{1}{p_1}dx,
 end{algn}
 and
begn{algn}
 nt^{nfty}_{gamma}frac{1}{p_1}dx  frac{1}{g(gamma)}nfty.
 end{algn}
 Ths s a contradcton. $blacksquare$[baselneskp]
 The proof of the followng theorem can be obtaned from the proof of Theorem 3.3.1.
 [baselneskp]
 {bf Theorem 3.3.2} textt{If $nt^{nfty}_{alpha}frac{1}{p_1(x)}dx  nfty$ and $nt^{nfty}
 _{alpha} p_0(x)dxnfty$ then (8) s nonoscllatory at $nfty$.}
 subsecton{Levn's Comparson Theorem}%Subsecton 4
 ndent Levn consders the followng equatons
 begn{algn}%Eq.s (12) and (13)
 u''+p_0(x)u=0,
 v''+q_0(x)v=0,
 end{algn}
 respectvely, where $xn[alpha,beta]$, $p_0,q_0$ are some contnuous functons on $[alpha,
 beta]$, $alpha,betanmathbb{R}$.
 ndent The method used by Levn converts the dfferental equatons (12) and (13) nto the 
Rccat equatons
 begn{algn}%Equatons (14) and (15)
 w'(x) = &w^2(x) + p_0(x),
 h'(x) = &h^2(x) + q_0(x),
 end{algn}
 where $xn[alpha,beta]$. Indeed, substtutng $w(x)=-frac{u'(x)}{u(x)}$ and $h(x)=-frac{v'(x)}
 {v(x)}$ nto the equatons (14) and (15), respectvely, then the equatons (12) and (13) are 
obtaned.[baselneskp]
 {bf Theorem 3.4.1} textt{Let $u(x)$ and $v(x)$ be nontrval solutons of (12) and (13), 
respectvely, such that $u(x)$ does not vansh on $[alpha,beta]$, $v(alpha)neq 0$ and the 
nequalty }
 begn{equaton}%Equaton (16)-frac{u'(alpha)}{u(alpha)}+nt^{x}_{alpha} p_0(t)dtbgg-frac{v'(alpha)}{v(alpha)}+nt^{x}
 _{alpha}q_0(t)dtbgg
 end{equaton} 
textt{holds for all $xn[alpha,beta]$. Then $v(x)$ does not vansh on $[alpha,beta]$ and the 
nequalty}
 begn{equaton}%Equaton (17)-frac{u'(x)}{u(x)}bgg-frac{v'(x)}{v(x)}bgg
 end{equaton}
 textt{holds, where $xn[alpha,beta]$. }[baselneskp]
 {bf Proof} If $u(x)$ does not vansh then $w(x)=-frac{u'(x)}{u(x)}$ s contnuous on $[alpha,
beta]$. Integratng the equaton (14) on $[alpha,x]$ one gets that
 begn{algn}nonumber
 w(x)-w(alpha)=nt^{x}_{alpha}w^2(t)dt+nt^{x}_{alpha}p_0(t)dt
 end{algn}
 or
 begn {equaton}
 w(x)=w(alpha)+nt^{x}_{alpha}w^2(t)dt+nt^{x}_{alpha}p_0(t)dt. %Eq.(18)
 end{equaton}
 By the hypotess (16), t can be wrtten as 
begn {equaton}
 w(x)ge-frac{u'(alpha)}{u(alpha)} + nt^{x}_{alpha}p_0(t)dt 0.
 end{equaton}
 Snce $v(alpha)neq0$ then $h(x)=-frac{v'(x)}{v(x)}$ s contnuous on $[alpha,gamma]$, 
where $alphagammaleqbeta$. Integratng the equaton (15) on $[alpha,gamma]$ where 
$x=gamma$ we get
 begn{algn}
 h(x)-h(alpha)=nt^{x}_{alpha}h^2(t)dt + nt^{x}_{alpha}q_0(t)dt,
 end{algn} 
or
 begn{algn}
 h(x)= h(alpha)+nt^{x}_{alpha}h^2(t)dt + nt^{x}_{alpha}q_0(t)dt.%Eq.(17)
 end{algn} 
From (19) and we have 
begn{algn}
 h(x)&ge-frac{v'(alpha)}{v(alpha)} + nt^{x}_{alpha}q_0(t)dt  &-w(alpha)-nt^{x}_{alpha}
 p_0(t)dtge -w(x),
 end{algn}
 where $xn[alpha,gamma]$. Consequently, $w(x)-h(x)$. In order to show that 
begn{equaton}
 h(x)w(x)%Eq(18)
 end{equaton}
 on $alphaleq$$x$$leqgamma$, t s suffcent to show that $w(x)h(x)$ on $alphaleq $$x$$ 
leqgamma$.
 ndent If we suppose to the contrary, there exsts an $x_0$ on the nterval $[alpha,gamma]$ 
such that the nequalty $w(x_0)leq h(x_0)$ holds. Then snce $w(alpha)h(alpha)$ from (17) 
(wth $x=alpha)$ and $w(x)$, $h(x)$ are contnuous on the nterval $[alpha,gamma]$, there 
exsts some pont $x_1$ n $(alpha,x_0]$ such that $w(x_1)=h(x_1)$ and $h(x)w(x)$ for $
 alphaleq$$x$$x_1$. Snce $w(x)-h(x)$ was establshed prevously, t follows that $h(x)
 w(x)$ for $alphaleq$$x$$x_1$. Consequently,  
begn{equaton}
 nt^{x_1}_{alpha}h^2(t)dtnt^{x_1}_{alpha}w^2(t)dt.
end{equaton}
 Hence, t s obtaned that
 begn{algn}
 h(x_1) &= h(alpha) + nt^{x_1}_{alpha}h^2(t)dt + nt^{x_1}_{alpha}q_0(t)dt & w(alpha) + 
nt^{x_1}_{alpha}w^2(t)dt + nt^{x_1}_{alpha} p_0(t)dt = w(x_1)
 end{algn}
 from equatons (18) and (19). However, ths s a contradcton.hspace{2cm}$blacksquare$
 [baselneskp]
 {bf Theorem 3.4.2}textt { Let $u(x)$ and $v(x)$ be two solutons of (12) and (13), respectvely, 
such that $u(x)$ does not vansh on $[alpha,beta]$, $v(beta)neq0$ and the nequalty}
 begn{equaton}%Equaton (19)
 frac{u'(beta)}{u(beta)}+nt^{beta}_{x} p_0(t)dtbggfrac{v'(beta)}{v(beta)}+nt^{beta}
 _{x}q_0(t)dtbgg
 end{equaton} %Eq.(20)
 textt {holds for all $xn[alpha,beta]$. Then $v(x)$ does not vansh on $[alpha,beta]$, and 
the nequalty}
 begn{equaton}%Eq.19
 frac{u'(x)}{u(x)}bggfrac{v'(x)}{v(x)}bgg, hspace{1cm} alphaleq xleqbeta.
 end{equaton}
 textt{holds}.[baselneskp]
 {bf Proof} If $u(x)$ does not vansh, $w(x)= -frac{u'(x)}{u(x)}$ s contnuous on $[alpha,beta]
 $ and ntegratng the equaton (14) and (15) on the nterval $[x,beta]$. We obtan that
 begn{equaton}
 w(beta) - w(x)= nt^{beta}_{x}w^2(t)dt + nt^{beta}_{x}p_0(t)dt
 end{equaton}
 and
 begn{equaton}
 h(beta) - h(x)= nt^{beta}_{x}h^2(t)dt + nt^{beta}_{x}q_0(t)dt.
 end{equaton}
 So t s found that
 begn{algn}-w(x) &geq -w(beta) + nt^{beta}_{x}p_0(t)dt &-h(x) geq -h(beta) + nt^{beta}_{x}
 q_0(t)dt.
 end{algn}
 Consequently, $-w(x)h(x)$ on $xleq x leqbeta$. Let us suppose the contrary. Then there 
exsts an $x_0$ on $[x,beta]$ such that $-w(x_0)leq -h(x_0)$. Then snce $-w(beta)
 h(beta)$ from hypotess (22) (wth $x=beta$) and snce $w(x)$ and $h(x)$ are contnuous on $
 [x,beta]$, there exsts $x_1$ n $[x_0,beta)$ such that 
begn{equaton}-w(x_1)=-h(x_1)
 end{equaton} 
and
 begn{equaton}-w(x)-h(x)
 end{equaton} 
for $x_1xleqbeta$. Snce $-w(x)-h(x)$ we also have
 begn{algn}-h(x_1) &= -h(beta) + nt^{beta}_{x_1}h^2(t)dt + nt^{beta}_{x_1}q_0(t)dt & -w(beta) + 
nt^{beta}_{x_1}w^2(t)dt + nt^{beta}_{x_1} p_0(t)dt = -w(x_1).
 end{algn}
 However ths s a contradcton. hspace{7.5cm}$blacksquare$[baselneskp]
 {bf Theorem 3.4.3}textt{ Suppose that there exsts a nontrval soluton $v(x)$ of the equaton 
(13) satsfyng the condtons $v(alpha)=v(beta)=v'(gamma)=0$, where $
 alphagammabeta$. If the nequaltes}
 begn{equaton}
 nt^{gamma}_{x} p_0(t)dt geq bggnt^{gamma}_{x} q_0(t)dtbgg, hspace{0.5cm} nt^{x}
 _{gamma} p_0(t)dt geq bggnt^{x}_{gamma} q_0(t)dtbgg
 end{equaton}
 textt {hold for all $x$ on $[alpha,gamma]$ and $[gamma,beta]$, then every soluton $u$ of 
the equaton (12) has at least one zero on $[alpha,beta]$. }[baselneskp]
 {bf Proof} Let $u$ be a nontrval soluton of (12) satsfyng $u'(gamma)=0$. Levn asserted 
that $u$ has at least one zero n each nterval $[alpha,gamma)$ and $(gamma,beta]$. Frst, 
observe that $u(gamma)neq0$. Otherwse, $u$ would be a trval soluton of (12). If $u$ had no 
zero n $(gamma,beta]$ then $u$ would have no zero on $[gamma,beta]$. Theorem 3.4.1 
supports the hypothess (21). Thus $v$ would have no zero on $[gamma,beta]$. Ths 
contradcts the hypothess $v(beta)=0$. Lkewse, f $u$ had no zero on $[alpha,gamma)$, 
Theorem 3.4.1 gves the contradcton that $v$ has no zero on $[alpha,gamma]$. Indeed, Levn 
has shown that $u$ has at least two zeros on $[alpha,beta]$, and hence every soluton of (12) 
has at least one zero on $[alpha,beta]$ by Sturm Seperaton Theorem.hspace{11.6cm}$
 blacksquare$
 secton{Some Crtera}
 ndentpar In ths secton we wll ntroduce some crtera on dsconjugacy, exstence of zeros of 
solutons and oscllaton.
 subsecton{Kreth's Crtera}
 ndent Kreth ntroduced the followng theorem on dsconjugacy.[baselneskp]
 {bf Theorem 4.1.1}textt{ If the dfferental equaton (9) s dsconjugate on $[alpha,beta]$, 
then}
 begn{equaton}%Eq.(24)
 nt^{beta}_{alpha} (q_1 eta'^2 + q_0 eta^2)dx 0,
 end{equaton}
 textt{for all admssable $eta(x)notequv 0$.}[baselneskp]
{bf Proof} If the equaton (9) s dsconjugate on $[alpha,beta]$, then there exsts a soluton 
$v(x)$ of (9) whch s dfferent from zero on $[alpha,beta]$. Usng The Pcone dentty wth 
$p_1equv q_1$ the followng can be wrtten 
begn{equaton}%Eq. (25)
 frac{d}{dx}bgg[frac{eta}{v}bgg(eta' q_1v-eta q_1 v'bgg)bgg] = frac{d}{dx}bgg(eta' eta 
q_1-frac{eta^2 q_1 v'}{v}bgg).
 end{equaton} 
Therefore, proceedng the steps n equaton (25) t s obtaned that 
begn{algn}
 &(eta' q_1)'eta + eta'^2 q_1-bgg[(q_1v')'frac{eta^2}{v} + (q_1v')bgg(frac{2etaeta'v
eta^2v'}{v^2}bgg)bgg]
 =&(eta' q_1)'eta + eta'^2 q_1 - q_0eta^2 - frac{2q_1v'etaeta'}{v} + frac{q_1eta^2 v'^2}
 {v^2}
 =& (eta' q_1)'eta - q_0eta^2 +q_1bgg(eta'-frac{eta v'}{v}bgg)^2=(eta' q_1)'eta + 
q_1eta'^2-frac{d}{dx}bgg(frac{q_1eta^2 v'}{v}bgg).
 end{algn}
 Cancellng $(eta' q_1)'eta$ and multplyng both sdes of the equaton by -1 we get
 begn{equaton}%Eq.(26)
 q_1eta'^2 + q_0eta^2 = q_1bgg(eta'-frac{eta v'}{v}bgg)^2 + frac{d}{dx}
 bgg(frac{q_1eta^2 v'}{v}bgg).
 end {equaton}
 Integratng both sdes of the equaton (26) on $[alpha,beta]$ t s found that
 begn{algn}
 nt^{beta}_{alpha}(q_1eta'^2+q_0eta^2)dx = nt^{beta}_{alpha} q_1bgg(eta'-frac{eta v'}
 {v}bgg)^2+ bgg(frac{q_1eta^2 v'}{v}bgg)bgg^{beta}_{alpha}.
 end{algn}
 Snce $eta(alpha)=eta(beta)=0$,
 begn{equaton}
 bgg(frac{q_1eta^2 v'}{v}bgg)bgg^{beta}_{alpha} = 0.
 end{equaton}
 So, we get 
begn{algn}
 nt^{beta}_{alpha}(q_1eta'^2+q_0eta^2)dx = nt^{beta}_{alpha} q_1bgg(eta'-frac{eta v'}
 {v}bgg)^2 ge 0
 end{algn}
 wth equalty f and only f 
begn{algn}
 eta'-frac{eta v'}{v} &=frac{eta' v - v'eta}{v}
 &=vbgg(frac{eta'}{v}-frac{v'eta}{v^2}bgg)
 &=0
 end{algn}
holds for the soluton $v0$. Then $(frac{eta}{v})'=0$ such that $frac{eta}{v}= C$, $C$ s a 
arbtrary constant. Ths gves that $eta=Cv$. Ths s a contradcton. Consequently, equaton 
(24) s establshed for all admssble $eta(x)notequv0$ on the nterval $[alpha,beta]$. 
hspace{4cm}  $blacksquare$ [baselneskp]
 {bf Corollary 4.1.1}textt{ If there exsts an admssble $eta(x)notequv0$ such that }
 begn{equaton}
 nt^{beta}_{alpha} (q_1 eta'^2 + q_0 eta^2)dxleq0,
 end {equaton}
 textt{then every soluton of the equaton (9) has a zero n $[alpha,beta)$. }
 subsecton{Leghton's Crteron}
 ndent Leghton has noted that Sturm-Louvlle Comparson Theorem s a specal case of 
Corollary 4.1.1. Indeed, f the followng nequalty holds
 begn{equaton}%Eq.(27)
 nt^{beta}_{alpha}[(p_1 - q_1)eta'^2 dx + (p_0-q_0)eta^2]dx leq nt^{beta}_{alpha} (q_1 
eta'^2 + q_0 eta^2)dx 
end{equaton}
 the followng s obtaned.[baselneskp]
 {bf Corollary 4.2.1}textt{ If there exsts an admssble $eta$ such that the nequalty (27) s 
satsfed, then every solutons of (9) has a zero n $[alpha,beta)$. }
 subsecton{Wntner's Crtera}
 ndentpar A.Wntner ntroduced some crtera on the oscllatory property of the equaton
 begn{equaton}%Eq.(28)
 y''+q(x)y=0
 end{equaton}
 where $q(x)$ s real valued, contnuous functon for large postve $x$, where $x_0leq xnfty$ 
and $x_0nmathbb{R}$.[baselneskp]
 {bf Theorem 4.3.1}textt{ The equaton (28) s oscllatory at $nfty$ f}
 begn{equaton}
 Q(x) = nt^{x}q(s)ds
 end{equaton}
 textt{satsfes}
 begn{equaton}%Eq.(29)
 lm_{xtonfty}frac{nt^{x}Q(s)ds}{x}=nfty.
 end{equaton}[baselneskp]
 {bf Proof} Let us assume the contrary. So that the equaton (28) s nonoscllatory. If $y(x)$ s a 
nontrval soluton of (28), then for large $x$ $y(x)$ s postve. Snce $y(x)$ does not vansh on 
some nterval around nfnty, t has a logartmc dervatve so that $h=frac{y'}{y}$. It satsfes the 
Rccat equaton
 begn{equaton}%Eq.(30)
h'=-q-h^2.
 end{equaton}
 Indeed one has
 begn{algn}
 h'&= frac{y''y-y'^2}{y^2}
 &=frac{y''y}{y^2}-frac{y'^2}{y^2}
 &=frac{-qy^2}{y^2}-frac{y'^2}{y^2} = -q-h^2.
 end{algn}
 The equaton (30) mples the nequalty $h'leq -q$ or $-h'geq q$. Integratng the nequalty 
$qleq -h'$ twce on some ntervals whose upper bounds are $x$, t s obtaned that
 begn{algn}
 &nt^{x}q(s)ds leq -h(x)+c_1
 end{algn}
 and
 begn{algn}
 & nt^{x}Q(s)ds leq -nt^{x}h(s)ds+c_1x+stackrel{sm}{c_2}
 end{algn}
 where $c_1,stackrel{sm}{c_2}$ are some constants. Therefore,
 begn{algn}
 nt^{x}Q(s)ds leq -lny(x)+c_1x+c_2
 end{algn}
 or
 begn{algn}%eq.(31)
 nt^{x}Q(s)ds - c_1x-c_2 leq - ln y(x).
 end{algn}
 Multplyng both sdes of the nequalty (31) by $1over x$, we get
 begn{algn}nonumber
 frac{nt^{x}Q(s)ds}{x} - c_1-frac{c_2}{x} leq frac{-lny(x)}{x}.
 end{algn}
 If hypotess (29) holds, then 
begn{equaton}
 lm_{xtonfty} ln y(x)=-nfty.
 end{equaton} 
Hence, ths gves a contradcton.hspace{7.3cm}$blacksquare$[baselneskp]
 {bf Theorem 4.3.2}textt{ The equaton (12) s nonoscllatory at $nfty$ f and only f there exsts 
a functon $v$ so that $v$ and $v'$ are contnuous satsfyng the nequalty}
 begn{equaton}
 v'(x) + v^2(x)leq -c(x)
 end{equaton}
 textt{for suffcently large $x$.}[baselneskp]
 {bf Proof} If equaton (12) s nonosclatory at $nfty$ and $u(x)$ s nontrval soluton of the 
equaton (12), there exsts some number $x_0$ such that $u(x)$ has no zero on $[x_0,nfty)$ 
and $v=frac{u'}{u}$ satsfes the Rccat equaton 
begn{equaton}
 v'(x) + v^2(x)= -c(x).
 end{equaton}
 Conversely, f there exsts a fucton $v$ satsfyng 
begn{equaton}-C(x)equv v'(x) + v^2(x)leq -c(x), xgeq x_0.
 end{equaton}
 Then
 begn{equaton}
 u(x)=expbgg[nt^{x}_{x_0} v(t) dtbgg]
 end{equaton}
 satsfes $u''+C(x)u=0$. Indeed
 begn{algn}
 u'(x)= &v(x)expbgg[nt^{ x}_{x_0}v(t)dtbgg],
 u''(x)= &v'(x)expbgg[nt^{ x}_{x_0}v(t)dtbgg]+v^2(x)expbgg[nt^{x}_{x_0}v(t)dtbgg] -&[v'(x)+v^2(x)]expbgg[nt^{x}_{x_0}v(t)dtbgg]=0.
 end{algn}
 Snce $c(x)leq C(x)$ on $x_0leq x nfty$, t follows from Sturm's Theorem 2.1 that no soluton 
of $u''+c(x)u=0$ can have more than one zero on $[x_0,nfty)$. hspace{9.5cm}$blacksquare$
 subsecton{Hartman's Crteron}
 ndent
 {bf Theorem 4.4.1} textt{ Every nonoscllatory equaton (12) has a soluton $u$ such that}
 begn{equaton}
 nt^{nfty} u^{-2}(t)dtnfty
 end{equaton}
 textt{s fnte and a nontrval soluton $v$ such that}
 begn{equaton}
 nt^{nfty} v^{-2}(t)dt = nfty.
 end{equaton}
 {bf Proof} Let $w$ be a soluton of the equaton (12) such that $w(x)0$ for $xx_0$, and 
Hartman defnes a second soluton by
 begn{equaton}%Eq.(36)
 u(x)= w(x)nt^{x}_{x_0}w^{-2}(t)dt, xx_0.
 end{equaton}
 Takng the dervatve of the equaton (36) twce, t s obtaned
 begn{algn}nonumber
 u'(x)&= w'(x)nt^{x}_{x_0}w^{-2}(t)dt + w^{-1}(x),
 end{algn}
and
 begn{algn}
 u''(x)&= w''(x)nt^{x}_{x_0}w^{-2}(t)dt.%Eq(37)
 end{algn}
 Addng $p_0(x)u(x)$ to both sdes of the equaton (37) we get
 begn{algn}
 0 = w''(x)nt^{x}_{x_0}w^{-2}(t)dt + p_0(x)w(x)nt^{x}_{x_0} w^{-2}(t)dt.
 end{algn}
 or
 begn{algn}
 0 = (w''(x)+ p_0(x)w(x))nt^{x}_{x_0}w^{-2}(t)dt,
 end{algn}
 where $w0$ on $[x_0,x]$. Then 
begn{equaton}
 w''(x)+ p_0(x)w(x)=0.
 end{equaton}
 The Wronskan of $u$ and $w$ s
 begn{algn}
 u(x)w'(x)-w(x)u'(x)&=w'(x)w(x) nt^{x}_{x_0}w^{-2}(t)dt &-w(x)bgg[w'(x)nt^{x}_{x_0}w^{-2}
 (t)dt+w^{-1}(x)bgg]
 &=-1.
 end{algn}
 Snce $u(x)0$ for $xx_0$, t follows that the rato $w(x)u(x)$ s dfferentable for $xx_0$ and 
begn{equaton}
 bgg(frac{w}{u}bgg)' = frac{w'u-u'w}{u^2} = frac{-1}{u^2}.
 end{equaton}
 Hence for $x_0x_1leq xnfty$, ntegratng $(frac{w}{u})'$ on $[x_1,x]$ one gets that
 begn{equaton}
 frac{w(x)}{u(x)}= frac{w(x_1)}{u(x_1)}-nt^{x}_{x_1} frac{dt}{u^{2}(t)}.
 end{equaton}
 If
 begn{equaton}
 lm_{xtonfty}nt^{x}_{x_1} frac{dt}{u^{2}(t)} = nfty
 end{equaton}
 then $w(x)u(x)$ would approach $-nfty$, contrary to fact that both $u(x)$ and $w(x)$ are 
postve for $xgeq x_1$.
 ndent To prove the second asserton of the Theorem 4.4.1, defne another soluton by
 begn{equaton}
 v(x)=u(x) nt^{nfty}_{x}frac{dt}{u^{2}(t)}, xx_0.
 end{equaton}  
Then $v(x)u(x) rghtarrow 0$ as $x rghtarrow nfty$ snce $u^{-2}$ s ntegrable on $
(x_0,nfty)$. From the dentty 
begn{algn}
 bgg(frac{u}{v}bgg)'&= frac{u'v-v'u}{v^2} 
&=frac{u'unt^{nfty}_{x}frac{dt}{u^2}-uu'nt^{nfty}_{x}frac{dt}{u^2}+u^2frac{1}{u^2}}
 {v^2}
 &=frac{1}{v^2}
 end{algn}
 t s obtaned that
 begn{algn}
 frac{u(x)}{v(x)}= frac{u(x_1)}{v(x_1)}+nt^{x}_{x_1}frac{dt}{v^2(t)}, x_0leq x_1leq x,
 end{algn}
 and hence 
begn{equaton}
 lm_{xtonfty}nt^{x}_{x_1} frac{dt}{v^{2}(t)} = nfty.
 end{equaton}
 Ths completes the proof.hspace{8cm} $blacksquare$[baselneskp]
 secton{Concluson}
 ndentpar In ths project our man am s to clarfy the very well known results on oscllaton and 
nonoscllaton (dsconjugacy) of some second-order Sturm-Louvlle equatons defned on some 
fnte and nfnte ntervals. As s known such equatons arse, for nstance, n the study of 
harmonc standng wave, vbratons of elastc bars and vbratng membranes. Therefore ths 
project may help the students n understandng some real-world problems n detal.
 secton{Referances}
 [1]C. A. Swanson, Comparson and Oscllaton Theory of Lnear Dfferental Equatons, Academc 
Press, 1968.[0.5cm]
 [2]E. Hlle, Lectures on Ordnary Dfferental Equatons, 1968.  [0.5cm]
 [3]K. Kreth, Oscllaton Theory, Sprnger-Verlag, 1973. [0.5cm]
 [4]M. A. Namark, Lnear Dfferental Operators Part II Lnear Dfferental Operators n Hlbert 
Space, George G. Harrap & Company, LTD. , 1968. [0.5cm]
 end{document}